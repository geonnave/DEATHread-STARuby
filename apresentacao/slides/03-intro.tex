\section{Introdução}

	\subsection{Problema escolhido}
%----------------------------------------------------------------------------%

	\begin{frame}
	\frametitle{Problema escolhido}
		 \begin{block}{}
				O problema escolhido é um simulador simples destinado a, obviamente, simular uma nave espacial em viajem. Os recursos principais da nave são separados em módulos, e cada módulo é executado por uma Thread.
		 \end{block}
	\end{frame}

	\subsection{Contexto}
	\begin{frame}
	\frametitle{Contexto}
		 \begin{block}{}
				\begin{itemize}
			    \item A nave possui uma classe que representa um sensor, com o status de recursos como energia, combustivel e dano.
					\pause
					\item Ao inicial a nave, os módulos são ligados, e cada Thread que representa a execução de um módulo começa a ser executada.
				\end{itemize}
		 \end{block}
	\end{frame}

\section{Threads no problema?}
	\subsection{Sobre Threads}
	\begin{frame}
	\frametitle{Por que usar Threads no problema?}
		 \begin{block}{}
		    Os modulos precisam notificar ao sensor o quanto estão gastando de recursos, ou se a nave recebeu algum dano, e aí se encontra a solução pelas Threads, já que todos devem notificar os sensores de forma concorrente.
		 \end{block}
	\end{frame}

	\begin{frame}
	\frametitle{Threads: explicações}
		 \begin{block}{}
				As Threads servem para separar a aplicação a ser executada em várias linhas de execução, de forma que estas linhas seja executadas concorrentemente e sincronizadamente.
		 \end{block}
	\end{frame}
	\begin{frame}
	\frametitle{Threads: explicações}
		 \begin{block}{}
				As Threads servem para separar a aplicação a ser executada em várias linhas de execução, de forma que estas linhas seja executadas concorrentemente e sincronizadamente.
		 \end{block}
	\end{frame}

	\subsection{Sobre Threads no ruby}
	\begin{frame}
	\frametitle{Threads no ruby}
		 \begin{block}{}
				No ruby, as Threads  são implementadas a nivel de aplicação, até mesmo porque é a maquina virtual do ruby que vai gerenciar a concorrencia das threads.
		 \end{block}
	\end{frame}

\section{Código principal}
	\subsection{Código principal}
	\begin{frame}
	\frametitle{Pseudo Código principal}
		 \begin{block}{}
				Como foi visto no  pseudocódigo, foi nescessário sincronizar o acesso aos sensores. Já que todos os modulos vão notificar os sensores concorrentemente, se isso não for sincronizado, dados não consistentes poderão ser colocados nos sensores.
		 \end{block}
	\end{frame}

	\begin{frame}
	\frametitle{Sobre o sistema}
		 \begin{block}{}
				Para tornar a notificação dos sensores Thread-safe (poder ser chamado por diversas threads sem causar inconsistencia nos dados), foi necessário implementar um semáforo em que cada thread passa por esse semáforo para notificar os sensores.
		 \end{block}
	\end{frame}

	\begin{frame}
	\frametitle{Sobre o sistema}
		 \begin{block}{}
				Para tornar a notificação dos sensores Thread-safe (poder ser chamado por diversas threads sem causar inconsistencia nos dados), foi necessário implementar um semáforo em que cada thread passa por esse semáforo para notificar os sensores.
		 \end{block}
	\end{frame}

	\begin{frame}
	\frametitle{Sobre o sistema}
		 \begin{block}{}
				O semáforo no caso, é uma classe chamada Mutex (mutual exclusion). Sendo assim, é criada uma instancia da classe Mutex, e essa instancia é chamada por cada Thread para sincronizar a parte devida do codigo.
		 \end{block}
	\end{frame}

	\section{Desenvolvimento}
	\subsection{Desenvolvimento}
	\begin{frame}
	\frametitle{Desenvolvimento}
		 \begin{block}{}
				O desenvolvimento foi feito na maior parte do tempo remotamente, compartilhando o espaço de desenvolvimento por meio de um repositorio no github,  com debates via messenger.
		 \end{block}
	\end{frame}

	\begin{frame}
	\frametitle{Sobre o sistema}
		 \begin{block}{}
				Os testes foram feitos basicamente por testes de unidade, assim, a cada classe criada, era testado a classe individualmente, e,  só posteriormente, integrar as classes para executar o problema.
		 \end{block}
	\end{frame}

	\section{Conclusão}
	\begin{frame}
	\frametitle{Conclusão}
		 \begin{block}{}
				O uso de Threads demonstrou algumas dificuldades, como por exemplo, entender como a maquina virtual faz a concorrencia das Threads; foi visto, por exemplo, que ao chamar uma Thread pelo metodo “run”, a aplicação não executa nenhuma outra Thread enquanto a execução desta não terminar, sendo nescessário usar o método “join” para repartir o tempo da CPU para diversas Threads que estão executando loops.
		 \end{block}
	\end{frame}


